\documentclass[a4paper]{article}
\usepackage[top=.8in, bottom=.8in, left=.8in, right=.8in]{geometry}
\usepackage[T1]{fontenc}
\usepackage{lmodern}
\usepackage{amssymb,amsmath}
\usepackage{ifxetex,ifluatex}
\usepackage{fixltx2e} % provides \textsubscript
% use microtype if available
\IfFileExists{microtype.sty}{\usepackage{microtype}}{}
\ifnum 0\ifxetex 1\fi\ifluatex 1\fi=0 % if pdftex
  \usepackage[utf8]{inputenc}
\else % if luatex or xelatex
  \usepackage{fontspec}
  \ifxetex
    \usepackage{xltxtra,xunicode}
  \fi
  \defaultfontfeatures{Mapping=tex-text,Scale=MatchLowercase}
  \newcommand{\euro}{€}
\fi
\ifxetex
  \usepackage[setpagesize=false, % page size defined by xetex
              unicode=false, % unicode breaks when used with xetex
              xetex]{hyperref}
\else
  \usepackage[unicode=true]{hyperref}
\fi
\hypersetup{breaklinks=true,
            bookmarks=true,
            pdfauthor={},
            pdftitle={},
            colorlinks=true,
            urlcolor=blue,
            linkcolor=magenta,
            pdfborder={0 0 0}}
\setlength{\parindent}{0pt}
\setlength{\parskip}{6pt plus 2pt minus 1pt}
\setlength{\emergencystretch}{3em}  % prevent overfull lines
\setcounter{secnumdepth}{0}

\author{}
\date{}

\begin{document}
\maketitle

\textsuperscript{1} Laboratory of Genetics, Department of Biology,
University of Turku, Turku, Finland. Email:
\href{mailto:mycalesis@gmail.com}{\texttt{mycalesis@gmail.com}}

\textsuperscript{2}

\textsuperscript{3}

\textsuperscript{4}

\subsection{Introduction}

Next Generation Sequencing (NGS) is considered a quantum leap in
improvement in techniques for DNA sequencing. The traditional Sanger
method {[}@sanger1977{]} allows sequencing around \textasciitilde{}1,000
bp per specimen in capillary-based machinery. This is almost meaningless
when compared to the sequencing output of NGS technology: 30 gigabases
of DNA in one single run {[}@reis2009{]}. This higher yield is achieved
by using massive parallel sequencing of PCR products based on DNA
systhensis on arrays of micron-scale beads on planar substrates (a
microchip) {[}@shendure2008{]}. As a result, millions of copies of
sequences (reads) are produced from the DNA templates. One application
of NGS is targeted sequencing of numerous loci of interest
{[}@ekblom2010{]} for several taxa, in one run, which is quicker and
cheaper than using the traditional Sanger method.

Indeed some studies have used NSG techniques to study phylogenetics at
the genomic level using miRNAs for higher level phylogeny in Arthropoda
{[}@campbell2011{]}. miRNAs are nonprotein coding RNAs of small length
involved in transcription {[}ref{]}

Research in Phylogenomics, using many more genes from genomes
{[}@wahlberg2008{]}, would be accelerated by using NGS due to the ease
to obtaining DNA data at massive scale. It would be very easy to
sequence many more than the 12 to 19 loci that so far have been used
phylogenomic studies {[}@wahlberg2008; @regier2013{]}.

However, researchers have been relying on the Sanger method for
sequencing a handful of genes to be used in phylogenetic inference in
Lepidoptera groups {[}@matos2013; @regier2013; @pena2011{]}.

One issue to develop is a way to obtain candidate genes suitable for
phylogenetic inference, i.e.~orthologous, single copy genes, lack of
introns, etc.

\begin{itemize}
\item
  why orthologous
\item
  why single copy
\item
  why no introns
\item
  why separated by xxxx distance
\end{itemize}

@regier2013 obtained their sequences from mRNA by performing reverse
transcription and PCR amplification {[}@regier2007{]}. mRNAs are
molecules transcribed from genomic DNA that have had introns spliced and
exons joined. Therefore, attempting to sequence these genes from genomic
DNA for other species will be troublesome due to the likely appearance
of introns. Intron sequences can be of various lengths across taxa and
would prove difficult to assess homology for phylogenetic studies.

@wahlberg2008 obtained candidate genes for phylogenomics by identifying
single copy and orthologus genes of \emph{Bombyx mori} from EST
libraries. EST sequences were searched in the \emph{Bombyx mori} genome
in order to identify suitable exons. These exon sequences were compared
against EST libraries of related Lepidoptera in order to obtain
homologous sequences for primer design.

\subsection{Other software}

CEPiNS {[}@hasan2013{]} is a software pipeline that uses predicted gene
sequences from both model and novel species to predict and identify
exons suitable for sequencing useful for phylogenetic inference.

\subsection{Exon models}

\subsubsection{Finding candidate genes from \emph{Bombyx mori}}

We need to obtain candidate genes to be used in phylogenetic inference
that have to fulfill the following requirements:

\begin{itemize}
\item
  Our genes should be orthologs.
\item
  Our genes should be single-copy genes.
\item
  Their sequence need to be around 251 DNA base pairs in length.
\end{itemize}

We will assume that our Next Generation Sequencer available is the
IonTorrent\_.

We have to consider the IonTorrent\_ platform requirements to arrive to
our target 250bp gene length:

\ctable[pos = H, center, botcap]{ll}
{% notes
}
{% rows
\FL
Primer & Length (bp)
\ML
Adapter A & 30
\\\noalign{\medskip}
5' Index & 8
\\\noalign{\medskip}
5' Degenerate Primer & 25
\\\noalign{\medskip}
Exon & \textbf{???}
\\\noalign{\medskip}
3' Degenerate Primer & 25
\\\noalign{\medskip}
3' Index & 8
\\\noalign{\medskip}
Adapter P & 23
\LL
}

For IonTorrent \url{http://www.iontorrent.com/} Platform 2, the maximum
length that can be sequenced is from 280bp to 320bp in total. Thus,
\texttt{320 - 119 = 201} is the maximum internal gene region (region
within degenerate primers).

Therefore, for the new set of primers, being designed for Platform2, we
have a maximum amplicon size of \texttt{201 + 25*2 = 251bp}.

The OrthoDB \url{ftp://cegg.unige.ch/OrthoDB6/} database has a catalog
of orthologous protein-coding genes for vertebrates, arthropods and
other living groups.

\subsection{Action items}

\subsubsection{Single Copy Genes}

From OrthoDB and get a list of single-copy genes for \emph{Bombyx mori}
Get the exon sequences from the CDS file for \emph{Bombyx} Use these CDS
sequences to extract the corresponding sequence in the \emph{Bombyx}
genome avoiding gaps so that we will only work with genes or gene
fragments that do not include introns (which is good if you want to do
phylogenetics).

\subsection{Exon Structure}

Comparison against other butterfly and moth genomes Exon validation
against genomes of \emph{Manduca}, \emph{Danaus} and \emph{Heliconius}

\subsection{Exon Aligment and Primer Design}

Need to get at least 4 sequences for primer desing using primers4clades
{[}@contreras2009{]}.

\subsection{Wet Lab}

\subsubsection{Multiplex PCR}

\subsubsection{Sample preparation for Next Generation Sequencing in
IonTorrent}

\subsection{Next Generation output analysis}

\subsubsection{find barcodes}

\subsubsection{find primers}

\subsubsection{quality control of reads}

\subsubsection{de novo assembly using velvet}

\subsubsection{alignment and storage in VoSeq?}

So that we can manage the high number of sequences and create datasets
for phylogenetic analysis very easily.

\subsection{Comparison with other methods}

@regier2009, @regier2008, @regier2013 use Reverse Transcription PCR from
mRNAs to avoid sequencing introns, although the corresponing genomic DNA
sequences are likely to include introns. Therefore if one use their
genes, it is not recommended to do ``direct gene amplification''
{[}@regier2007{]}.

\section{References}

Targeted sequencing {[}@godden2012{]}

\end{document}
