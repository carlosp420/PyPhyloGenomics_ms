\documentclass[]{article}
\usepackage[T1]{fontenc}
\usepackage{lmodern}
\usepackage{amssymb,amsmath}
\usepackage{ifxetex,ifluatex}
\usepackage{fixltx2e} % provides \textsubscript
% use microtype if available
\IfFileExists{microtype.sty}{\usepackage{microtype}}{}
\ifnum 0\ifxetex 1\fi\ifluatex 1\fi=0 % if pdftex
  \usepackage[utf8]{inputenc}
\else % if luatex or xelatex
  \usepackage{fontspec}
  \ifxetex
    \usepackage{xltxtra,xunicode}
  \fi
  \defaultfontfeatures{Mapping=tex-text,Scale=MatchLowercase}
  \newcommand{\euro}{€}
\fi
\usepackage{ctable}
\usepackage{float} % provides the H option for float placement
\ifxetex
  \usepackage[setpagesize=false, % page size defined by xetex
              unicode=false, % unicode breaks when used with xetex
              xetex]{hyperref}
\else
  \usepackage[unicode=true]{hyperref}
\fi
\hypersetup{breaklinks=true,
            bookmarks=true,
            pdfauthor={Carlos Peña*,1; Victor Solis2; Pável Matos3; Chris Wheat4},
            pdftitle={PyPhyloGenomics: toolkit and protocol for developing phylogenetic markers in novel species for Next Generation Sequence data},
            colorlinks=true,
            urlcolor=blue,
            linkcolor=magenta,
            pdfborder={0 0 0}}
\setlength{\parindent}{0pt}
\setlength{\parskip}{6pt plus 2pt minus 1pt}
\setlength{\emergencystretch}{3em}  % prevent overfull lines
\setcounter{secnumdepth}{0}

\title{PyPhyloGenomics: toolkit and protocol for developing phylogenetic
       markers in novel species for Next Generation Sequence data}
\author{Carlos Peña\textsuperscript{*,1} \and Victor Solis\textsuperscript{2} \and Pável Matos\textsuperscript{3} \and Chris Wheat\textsuperscript{4}}
\date{2013-05-08}

\begin{document}
\maketitle

\textsuperscript{1}Laboratory of Genetics, Department of Biology,
University of Turku, Turku, Finland

\textsuperscript{2}

\textsuperscript{3}Biology Centre AS CR, v.v.i., Institute of
Entomology, Ceske Budejovice, Czech Republic

\textsuperscript{4}Population Genetics, Department of Zoology, Stockholm
University, Stockholm, Sweden

*\textbf{Corresponding author:} E-mail:
\href{mailto:mycalesis@gmail.com}{\texttt{mycalesis@gmail.com}}

\section{Introduction}

Next Generation Sequencing (NGS) is considered a quantum leap in
improvement in techniques for DNA sequencing {[}@loman2012{]}. The
sequencing output of NGS technology is around 30 gigabases of DNA in one
single run {[}@reis2009{]} while the traditional Sanger method
{[}@sanger1977{]} allows sequencing only \textasciitilde{}1,000 bp per
specimen in the old capillary-based technology. This higher yield is
achieved by using massive parallel sequencing of PCR products based on
DNA synthesis using micron-scale beads on planar substrates (a
microchip) {[}@shendure2008{]}. As a result, millions of copies of
sequences (reads) are produced from the DNA templates. One application
of NGS is targeted sequencing of numerous loci of interest in one run
{[}@ekblom2010{]}, which is quicker and cheaper than using the Sanger
method.

Research in phylogenomics can be accelerated by using NGS due to the
ease to obtain DNA data at massive scale. It would be very easy to
sequence many more than the 12 to 19 loci that so far have been used in
phylogenomic studies {[}@wahlberg2008; @regier2013{]}. However,
researchers have been relying on the Sanger method for sequencing a
handful of genes to be used in phylogenetic inference in several
Lepidoptera groups {[}@matos2013; @regier2013; @pena2011{]}.

One issue to develop is a way to obtain candidate genes suitable for
phylogenetic inference, i.e.~orthologs, single copy genes, lack of
introns, etc. Ortholog genes are those that share a common ancestor
during their evolutionary history {[}@chiu2006{]} and can be considered
as homologous structures useful for comparative systematics.

Gene duplication is a common phenomenon in animals and plants
{[}@duarte2010{]} producing paralog genes with a degree of similarity
depending on the time of divergence since duplication. Paralogs are
problematic for phylogenetic inference and these are not normally used
because they can cause error and artifacts {[}@sanderson2002;
@fares2005{]}.

Eukaryotic genes contain introns, sequences that are discarded during
the process of protein synthesis {[}@page1998{]} and can vary widely in
size among different species {[}@carvalho1999{]}. Thus, it might be
difficult to assess homology base pair positions if the sequences vary
in length among the studied novel species. However, introns have been
useful in phylogenetic studies of certain organisms {[}e.g.
@prychitko1997; @fujita2004{]}.

\begin{itemize}
\item
  why separated by xxxx distance
\end{itemize}

Indeed some studies have used NSG techniques to study phylogenetics at
the genomic level using miRNAs for higher level phylogeny in Arthropoda
{[}@campbell2011{]}. miRNAs are nonprotein coding RNAs of small length
involved in DNA transcription and gene regulation. Using miRNAs for
phylogenetics has the drawback that these molecules are not easy to
sequence from genomic DNA as miRNAs are processed in the cell and
shortened to \textasciitilde{} 22 base pair sequences
{[}@wienholds2005{]}.

@regier2013 obtained nuclear gene sequences from mRNA by performing
reverse transcription and PCR amplification {[}@regier2007{]}. mRNAs are
molecules transcribed from genomic DNA that have had introns spliced and
exons joined. Therefore, attempting to sequence these genes from genomic
DNA for other species will be troublesome due to the likely appearance
of introns. Intron sequences can be of various lengths across taxa and
would prove difficult to assess homology for phylogenetic studies.

@wahlberg2008 obtained candidate genes for phylogenomics by identifying
single copy and orthologus genes of \emph{Bombyx mori} from EST
libraries. They searched for EST sequences in the \emph{Bombyx mori}
genome in order to identify suitable exons. These exon sequences were
compared against EST libraries of related Lepidoptera species in order
to obtain homologous sequences for primer design. Thus, this method
depends on the availability of EST sequences which are single reads of
cDNA that might contain numerous errors and are prone to artefacts
{[}@parkinson2002{]}.

According to @wahlberg2008, it is easier to employ genomic DNA for
phylogenetic practice due to several reasons: (i) genomic DNA does not
degrade so quickly as RNA; (ii) it is simpler to preserve in the field;
(iii) it can be sequenced even from dry material (for example museum
specimens); and (iv) it is the most commonly used DNA in molecular
systematics.

Thus, a method is needed to find candidate genes that can be easily
sequenced from genomic DNA across several lineages. One strategy to
fulfill this goal could be comparing genomic sequences of model species
and extract suitable genes that can be sequenced in novel species from
simple extractions of genomic DNA.

In this paper, we describe a protocol for finding genes from genomic DNA
that are suitable for phylogenomic studies. We describe the software
package PyPhyloGenomics, written in Python language, that includes
bioinformatic tools useful for automated gene finding, primer design and
NGS data analysis. We have used this software to find homologous exons
across genomes from several model organisms. Our software also includes
tools to filter output reads from NGS and assemble the sequences for
each specimen and their sequenced genes so that datasets can be
assembled for analysis in common software for phylogenetic inference.

\section{Methods}

\subsection{Finding candidate genes from \emph{Bombyx mori}}

We are interested in studing the phylogenetic relationships of lineages
in the Lepidoptera. Hence, we decided to use the \emph{Bombyx mori}
genome as starting point (although any genome can be used) to obtain
candidate genes suitable fo r sequencing across novel species. As
explained in the introduction, genes to be used in phylogenetic
inference have to fulfill the following requirements: (i) the genes
should be orthologs; (ii) the genes should be single-copy genes; (iii)
their sequence need to be around 251 DNA base pairs in length for easy
sequencing in our in-house Next Generation Sequencer, an Ion Torrent PGM
sequencer from Life Technologies (\url{http://www.iontorrent.com/}).

The OrthoDB database \url{ftp://cegg.unige.ch/OrthoDB6/} has a catalog
of orthologous protein-coding genes for vertebrates, arthropods and
other living groups. We parsed this list with the module OrthoDB from
our package PyPhyloGenomics and obtained a list of single-copy,
orthologous gene IDs for \emph{Bombyx mori} (12 167 genes in total).

A function in our module BLAST extracted the sequences for those genes
from the \emph{Bombyx mori} CDS sequences (available at
\url{http://silkdb.org}). We BLASTed the sequences against the
\emph{Bombyx mori} genome and discarded those containing introns. We
kept genes with sequences longer than 300bp in length, and separated by
at least 810kb so that they can be considered independent evolutionary
entities and obtained 575 exons.

We validated those exons by automated search of these exons in other
genomes of species in Lepidoptera, such as, \emph{Danaus},
\emph{Heliconius} and \emph{Manduca}. This search is automated by using
functions in our module BLAST that take as input the list of genes from
\emph{Bombyx mori}, and the file with genomic sequences for the target
species. During validation, PyPhyloGenomics creates FASTA format files
by appending matching sequences from the tested genomes. It also
automates the alignment of sequences by using the software MUSCLE.

PyPhyloGenomics contains functions to automatically design degenerate
primers from the homologous sequences by delivering the sequences to
primer4clades and receiving the designed primers. primer4clades is a web
service based on the CODEHOP strategy for primer design
{[}@contreras2009{]}.

It is recomended that both alignment and designed primers to be analyzed
carefully to make sure that the are no problems. After this step, one
can have around XXX genes ready to be sequenced across novel species in
Lepidoptera for many species if NGS techniques are used.

\subsubsection{Sample preparation for Next Generation Sequencing in Ion
Torrent}

We followed the library preparation protocol for NGS by @meyer2010 with
minor modifications for the Ion Torrent technology. This method consists
in attaching and index (or barcode) to the amplified PCR products of
each specimen previous to sequencing. Therefore, it will be possible to
separate reads from the NGS data according to index.

We sequenced several individuals of a wide range of species in the
Lepidoptera. We also sequenced specimens of the model species
\emph{Bombyx mori}, \emph{Danaus?}, (codes \texttt{XXX}) as control
samples in order to validate our NGS data assembly protocols.

The Ion Torrent platform 2 can sequence from 280 to 320bp per read. The
Ion Torrent adapter, index and primer sequences make around 119 base
pairs in length, leaving around 201 bp as the maximum internal gene
region that can be sequenced (region within degenerate primers) (Table
1). This is the region per gene (or exon) that is potentially
informative for phylogenetic inference.

\textbf{Table 1.} Adaptors and primers needed for sequencing in the NGS
Ion Torrent platform 2. The maximum length of sequenced amplicon is
\textasciitilde{} 201 bp after discarding primer regions.

\ctable[pos = H, center, botcap]{lr}
{% notes
}
{% rows
\FL
Primer & Length (bp)
\ML
Adapter A & 30
\\\noalign{\medskip}
5' Index & 8
\\\noalign{\medskip}
5' Degenerate Primer & 25
\\\noalign{\medskip}
Exon & \textbf{119}
\\\noalign{\medskip}
3' Degenerate Primer & 25
\\\noalign{\medskip}
3' Index & 8
\\\noalign{\medskip}
Adapter P & 23
\LL
}

\subsection{Next Generation output analysis}

The raw output data of the Ion Torrent was a FASTQ format file of XXX
MB? and XXX short reads up to XX bp in length. We created a BLAST
database with the exon sequences of candidate genes found after the exon
validation of \emph{B. mori} genes across the genomes of the model
Lepidoptera species. We used blasted the find NGS reads against this
database in order to find those matching the homologous regions of the
candidate genes. All reads were separated in bins according to the match
against candidate genes.

We separated reads from each bin according to each specimen index (or
barcode). The sequencing process produced many reads with errors in the
barcode section. Thus, we measured the Levenshtein distance between the
sequenced index region and our indexes in order to measure the number of
nucleotide changes needed to convert one index into the other. We
assumed indexes to be the same if the Levenshtein distance was smaller
than 2 units (as our indexes differ in two or more nucleotides). Our
module XXX in PyPhyloGenomics is able to do the separation according to
indexes taking into account Levenshtein distances and compare the
forward and reverse complement of the index sequences.

We performed quality control of the reads using the software
fastx\_tools and assembled consensus sequences for each bin containing
reads for specimen using the \emph{velvet} assembler {[}@zerbino2008{]}.

Our functions in PyPhyloGenomics automate this process and require as
input the parameters needed for triming low quality reads, triming of
indexes and coverage threshold for assembly in velvet.

The output file is a FASTA format file containing the assembled
sequences per specimen and gene.

It is recommended to manually check the assembled sequences to discard
errors and spurious sequences.

We uploaded all our sequences to our molecular database software VoSeq
{[}@pena2012{]} for creation of datasets to be used in phylogenetic
analysis later on.

\subsection{Comparison with other methods}

@regier2009, @regier2008, @regier2013 use Reverse Transcription PCR from
mRNAs to avoid sequencing introns, although the corresponing genomic DNA
sequences are likely to include introns. Therefore if one use their
genes, it is not recommended to do ``direct gene amplification''
{[}@regier2007{]}.

\subsection{Other software}

CEPiNS {[}@hasan2013{]} is a software pipeline that uses predicted gene
sequences from both model and novel species to predict and identify
exons suitable for sequencing useful for phylogenetic inference.

\section{References}

Targeted sequencing {[}@godden2012{]}

\end{document}
